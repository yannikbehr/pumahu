%% BioMed_Central_Tex_Template_v1.06
%%                                      %
%  bmc_article.tex            ver: 1.06 %
%                                       %

%%IMPORTANT: do not delete the first line of this template
%%It must be present to enable the BMC Submission system to
%%recognise this template!!

%%%%%%%%%%%%%%%%%%%%%%%%%%%%%%%%%%%%%%%%%
%%                                     %%
%%  LaTeX template for BioMed Central  %%
%%     journal article submissions     %%
%%                                     %%
%%          <8 June 2012>              %%
%%                                     %%
%%                                     %%
%%%%%%%%%%%%%%%%%%%%%%%%%%%%%%%%%%%%%%%%%


%%%%%%%%%%%%%%%%%%%%%%%%%%%%%%%%%%%%%%%%%%%%%%%%%%%%%%%%%%%%%%%%%%%%%
%%                                                                 %%
%% For instructions on how to fill out this Tex template           %%
%% document please refer to Readme.html and the instructions for   %%
%% authors page on the biomed central website                      %%
%% http://www.biomedcentral.com/info/authors/                      %%
%%                                                                 %%
%% Please do not use \input{...} to include other tex files.       %%
%% Submit your LaTeX manuscript as one .tex document.              %%
%%                                                                 %%
%% All additional figures and files should be attached             %%
%% separately and not embedded in the \TeX\ document itself.       %%
%%                                                                 %%
%% BioMed Central currently use the MikTex distribution of         %%
%% TeX for Windows) of TeX and LaTeX.  This is available from      %%
%% http://www.miktex.org                                           %%
%%                                                                 %%
%%%%%%%%%%%%%%%%%%%%%%%%%%%%%%%%%%%%%%%%%%%%%%%%%%%%%%%%%%%%%%%%%%%%%

%%% additional documentclass options:
%  [doublespacing]
%  [linenumbers]   - put the line numbers on margins

%%% loading packages, author definitions

%\documentclass[twocolumn]{bmcart}% uncomment this for twocolumn layout and comment line below
%\documentclass[doublespacing, linenumbers]{bmcart}
\documentclass[doublespacing]{bmcart}

%%% Load packages
%\usepackage{amsthm,amsmath}
\RequirePackage{natbib}
%\RequirePackage[authoryear]{natbib}% uncomment this for author-year bibliography
%\RequirePackage{hyperref}
\usepackage[utf8]{inputenc} %unicode support
%\usepackage[applemac]{inputenc} %applemac support if unicode package fails
%\usepackage[latin1]{inputenc} %UNIX support if unicode package fails
\usepackage{graphicx}


%%%%%%%%%%%%%%%%%%%%%%%%%%%%%%%%%%%%%%%%%%%%%%%%%
%%                                             %%
%%  If you wish to display your graphics for   %%
%%  your own use using includegraphic or       %%
%%  includegraphics, then comment out the      %%
%%  following two lines of code.               %%
%%  NB: These line *must* be included when     %%
%%  submitting to BMC.                         %%
%%  All figure files must be submitted as      %%
%%  separate graphics through the BMC          %%
%%  submission process, not included in the    %%
%%  submitted article.                         %%
%%                                             %%
%%%%%%%%%%%%%%%%%%%%%%%%%%%%%%%%%%%%%%%%%%%%%%%%%


%\def\includegraphic{}
%\def\includegraphics{}



%%% Put your definitions there:
\startlocaldefs
\newcommand{\bm}[1]{{\textbf{#1}}}
\endlocaldefs

%\newcommand{\bm}[1]{{\textbf{#1}}}
%%% Begin ...
\begin{document}

%%% Start of article front matter
\begin{frontmatter}

\begin{fmbox}
\dochead{Research}

%%%%%%%%%%%%%%%%%%%%%%%%%%%%%%%%%%%%%%%%%%%%%%
%%                                          %%
%% Enter the title of your article here     %%
%%                                          %%
%%%%%%%%%%%%%%%%%%%%%%%%%%%%%%%%%%%%%%%%%%%%%%

\title{Continuous estimates of heat emission at Mt. Ruapehu using
the Unscented Kalman Smoother}

%
%%%%%%%%%%%%%%%%%%%%%%%%%%%%%%%%%%%%%%%%%%%%%%
%%                                          %%
%% Enter the authors here                   %%
%%                                          %%
%% Specify information, if available,       %%
%% in the form:                             %%
%%   <key>={<id1>,<id2>}                    %%
%%   <key>=                                 %%
%% Comment or delete the keys which are     %%
%% not used. Repeat \author command as much %%
%% as required.                             %%
%%                                          %%
%%%%%%%%%%%%%%%%%%%%%%%%%%%%%%%%%%%%%%%%%%%%%%

\author[
   addressref={aff1},                   % id's of addresses, e.g. {aff1,aff2}
   corref={aff1},                       % id of corresponding address, if any
   email={y.behr@gns.cri.nz}   % email address
]{\inits{YB}\fnm{Yannik} \snm{Behr}}
\author[
   addressref={aff1},
   email={s.sherburn@gns.cri.nz}
]{\inits{SS}\fnm{Steve} \snm{Sherburn}}
\author[
   addressref={aff1},
   email={t.hurst@gns.cri.nz}
]{\inits{TH}\fnm{Tony} \snm{Hurst}}


%%%%%%%%%%%%%%%%%%%%%%%%%%%%%%%%%%%%%%%%%%%%%%
%%                                          %%
%% Enter the authors' addresses here        %%
%%                                          %%
%% Repeat \address commands as much as      %%
%% required.                                %%
%%                                          %%
%%%%%%%%%%%%%%%%%%%%%%%%%%%%%%%%%%%%%%%%%%%%%%

\address[id=aff1]{%
    \orgname{GNS Science},
  \street{114 Karetoto Rd},
  \postcode{3384}
  \city{Taupo},
  \cny{New Zealand}
}

%%%%%%%%%%%%%%%%%%%%%%%%%%%%%%%%%%%%%%%%%%%%%%
%%                                          %%
%% Enter short notes here                   %%
%%                                          %%
%% Short notes will be after addresses      %%
%% on first page.                           %%
%%                                          %%
%%%%%%%%%%%%%%%%%%%%%%%%%%%%%%%%%%%%%%%%%%%%%%

\begin{artnotes}
%\note{Sample of title note}     % note to the article
%\note[id=n1]{Equal contributor} % note, connected to author
\end{artnotes}

\end{fmbox}% comment this for two column layout

%%%%%%%%%%%%%%%%%%%%%%%%%%%%%%%%%%%%%%%%%%%%%%
%%                                          %%
%% The Abstract begins here                 %%
%%                                          %%
%% Please refer to the Instructions for     %%
%% authors on http://www.biomedcentral.com  %%
%% and include the section headings         %%
%% accordingly for your article type.       %%
%%                                          %%
%%%%%%%%%%%%%%%%%%%%%%%%%%%%%%%%%%%%%%%%%%%%%%

\begin{abstractbox}

\begin{abstract} % abstract
    Volcanic lakes often capture a significant amount of volcanic heat emission
    and thus provide a unique opportunity to monitor changes inside the volcano.
    We present a Bayesian inversion method to automatically infer volcanic heat
    emission at the base of a volcanic lake from lake monitoring data using a
    non-linear Kalman Smoother. Our method accounts for the, sometimes large,
    uncertainties in observations and the underlying physics-based model to
    generate probabilistic estimates of heat emission. We verify our results
    using a synthetic test case and then estimate the daily heat input rate into
    Mt. Ruapehu's Crater Lake, New Zealand, between 2016 and 2022.
    Time-frequency analysis of the heat input rate shows dominant periods of
    heating cycles ranged between 100 - 250 days. The period between 2017 and
    2020 was dominated by shorter cycles and greater-than-average heat input
    rate which points to changes in the magmatic heat supply and the
    hydrothermal system during this time.
\end{abstract}

%%%%%%%%%%%%%%%%%%%%%%%%%%%%%%%%%%%%%%%%%%%%%%
%%                                          %%
%% The keywords begin here                  %%
%%                                          %%
%% Put each keyword in separate \kwd{}.     %%
%%                                          %%
%%%%%%%%%%%%%%%%%%%%%%%%%%%%%%%%%%%%%%%%%%%%%%

\begin{keyword}
\kwd{Crater lakes}
\kwd{Non-linear Kalman Smoother}
\kwd{Volcanic heat emission}
\end{keyword}

% MSC classifications codes, if any
%\begin{keyword}[class=AMS]
%\kwd[Primary ]{}
%\kwd{}
%\kwd[; secondary ]{}
%\end{keyword}

\end{abstractbox}
%
%\end{fmbox}% uncomment this for twcolumn layout

\end{frontmatter}

%%%%%%%%%%%%%%%%%%%%%%%%%%%%%%%%%%%%%%%%%%%%%%
%%                                          %%
%% The Main Body begins here                %%
%%                                          %%
%% Please refer to the instructions for     %%
%% authors on:                              %%
%% http://www.biomedcentral.com/info/authors%%
%% and include the section headings         %%
%% accordingly for your article type.       %%
%%                                          %%
%% See the Results and Discussion section   %%
%% for details on how to create sub-sections%%
%%                                          %%
%% use \cite{...} to cite references        %%
%%  \cite{koon} and                         %%
%%  \cite{oreg,khar,zvai,xjon,schn,pond}    %%
%%  \nocite{smith,marg,hunn,advi,koha,mouse}%%
%%                                          %%
%%%%%%%%%%%%%%%%%%%%%%%%%%%%%%%%%%%%%%%%%%%%%%

%%%%%%%%%%%%%%%%%%%%%%%%% start of article main body
% <put your article body there>

%%%%%%%%%%%%%%%%
%% Background %%
%%
\section{Introduction}

The amount and rate of heat emitted by a volcano provides important clues about
volcanic processes at depth. For example, a sudden rise in the heat output rate
combined with increased microgravity signals may result from the emplacement of
fresh magma at shallow depth \cite{Brown1991}. A slow decrease in the heat
emission rate together with a high carbon-sulfur ratio of the emitted gas may
point to the formation of a hydrothermal seal
\cite{christensonCyclicProcessesFactors2010}.

Volcanic lakes help with estimating heat emissions. Often, the majority of a
volcano's heat and gas output passes through the crater lake. Crater lakes act thus
like a filter, integrating heat and gas input from the vents entering the lake.

Monitoring changes in temperature, water mass, and ion concentration in these
lakes can, therefore, serve as proxies for changes in gas- and steam-release by
the hydrothermal system and, ultimately, the melt zone beneath the volcano. 

Ruapehu Crater Lake (Te Wai \={a}-moe in Te Reo Maori and referred to as RCL from
hereon) on Mt. Ruapehu, an active andesitic stratovolcano in the center of New
Zealand's North Island (Figure \ref{overview}), is one such example with records
of observations dating back to the 19th century \cite{Friedlander1898}. Mt.
Ruapehu erupted at least 602 times since 1830 with many phreatic and
phreato-magmatic eruptions and two major magmatic episodes
\citep{Scott2013,HistoricEruptiveActivity2022}. Although the latest magmatic
event between June 1995 to November 1997 and a subsequent dam collapse changed
the shape and volume of the lake, it has been largely unchanged since then and
the lake's bathymetry is well known from times when it was empty.

First attempts to quantify the amount of heat entering the lake date back to
1966 \citep{Dibble1966}. Since 2016 water temperature and lake level have been
monitored continuously and water samples for ion concentration measurements have
been taken on an approximately monthly basis. This enables us to continuously
estimate heat input into the lake and, in principle, automatically detect
changes as soon as they occur.

We typically infer the heat input from a mass and energy balance calculation:
the total mass and energy flow into and out of the lake has to account for the
observed changes in lake temperature, water level and ion dilution
\cite{Hurst1981, Hurst1991, Stevenson1992, Fournier2009, Scott1994}. At RCL, and
probably at most other volcanic lakes, this is an underdetermined problem, that
is the number of unknowns exceeds the number of constraints. Hence, it is
necessary to constrain some of the parameters in the mass and energy balance
model using prior knowledge. Further, some observations and input parameters
come with large uncertainties and are sampled irregularly.

In the following, we propose a probabilistic (Bayesian) inference approach which
naturally handles these difficulties and transparently propagates uncertainties of
observations and prior assumptions into posterior uncertainties of heat input.

\section{Probabilistic inference method}\label{Pim}

\subsection{Physical model}\label{phm}

The mass and energy balance model can be written in terms of three coupled
ordinary differential equations \cite{Hurst1991, Stevenson1992}. The rate of
lake water mass change ($\dot{M}$) is the difference between the rate of steam
input through the vent system beneath the lake ($\dot{M_i}$), the water inflow
rate through melt and precipitation ($\dot{M_s}$), the outflow rate through
seepage and overflow ($\dot{M_o}$), and the rate of mass loss through
evaporation ($\dot{M_e}$): 
\begin{equation}\label{ode_M}
	\frac{dM}{dt} = \dot{M_i} + \dot{M_s} - \dot{M_o} - \dot{M_e} 
\end{equation}

The energy content of the lake can be described with the notion of enthalpy,
$H$, which is defined relative to a reference enthalpy. In the following we take
the enthalpy of the lake at temperature $T=0~^{\circ}C=T_0$ as reference
$H_0=H(T_0)$. At constant pressure, the enthalpy of the lake can then be written
as a function of $T$, $T_0$, $M$ and the specific heat of water, $c_w$:
\begin{equation}
	H(T) = H_0 + \int_{T_0}^{T}c_wM\tau d\tau
\end{equation}

For all following equations we take $c_w=constant$, such that the enthalpy of
lake water at temperature $T$ can be written as $H(T) = H_0 + c_wMT$.

The rate of enthalpy change of the lake water can then be defined as a function
of the rate of energy input from volcanic sources ($\dot{Q_i}$), the rate of
energy gain due to solar irradiation from short-wavelength radiation
($\dot{Q_r}$), the rate of energy loss through outflow/seepage
($c_wT\dot{M_o}$), the rate of surface energy losses due to evaporation (forced
and free convection), sensible heat, and long-wavelength radiation
($\dot{Q_e}$), and the rate of energy loss from heating incoming surface water
($c_wT\dot{M_s}$): 

\begin{equation}\label{ode_E}
    \frac{dH}{dt}=\frac{d(c_wTM)}{dt}=c_wT\dot{M} + c_wM\dot{T} =
	\dot{Q_i} + \dot{Q_r} - c_wT\dot{M_o} - \dot{Q_e} - c_wT\dot{M_s} 
\end{equation}
      
Because RCL is at an altitude of $\sim$~2500~m and big parts of its catchment
are covered by snow and ice all year round we assume the temperature of incoming
surface water to be at 0 $^{\circ}C$. Several empirical and theoretical equations exist
to describe the surface losses and we refer the reader to previous publications
for more detailed discussions \cite{Stevenson1992, hurstCraterLakeEnergy2015}.
There is no consensus on the best quantitative model of evaporation from warm
lakes and, in fact, the most suitable equation may depend on the location of and
available observations from the lake. Appendix \ref{A} describes the equations
used in this study in detail which include the recent development on forced
(i.e. wind-induced) convection \cite{sartoriCriticalReviewEquations2000}. It is
important to note here that $Q_e$ is a function of $T$ and the windspeed above
the lake. 

Following from Equation \ref{ode_E} the temperature change $\frac{dT}{dt}$ can
then be written as:
\begin{equation}\label{ode_T}
	\frac{dT}{dt}=\frac{1}{c_wM}\left(-\dot{Q_e} - c_wT\dot{M_o} + \dot{Q_{eff}} + \dot{Q_r}\right )
	-\frac{T}{M}\dot{M}
\end{equation}

Finally, the change in ion concentration expressed in total ion amount ($X$) of,
for example, $Mg^{2+}$ can be defined as:
\begin{equation}\label{ode_X}
	\frac{dX}{dt}=-\dot{M_o}\frac{X}{M}
\end{equation}

This equation is only true under the assumption that $X$ only changes due to
dilution and there is no re-supply between eruptive episodes. Equation
\ref{ode_X} gives an estimate of the average outflow over periods on the order
of months but does not capture shorter term variations in the outflow.

The main challenge in solving the mass and energy balance equations for
$\dot{Q_i}$ arises from the non-linear relationship between $Q_e$ and ${T}$
as shown in Figure \ref{sens}.

\subsection{Inference}\label{inf}

The objective of the inference problem is to estimate the rate at which heat
is entering the lake from the hydrothermal system below ($\dot{Q_i}$).
To constrain this we use the currently available observations of $T$, $M$,
$X$, and $\dot{M_o}$. Putting it more concisely, we are looking for 

\begin{equation}\label{bsm}
    p(\bm{x}_k|\bm{y}_{1:t})
\end{equation}

where $\bm{y}_{1:t}$ is the time series of $m$ observations and $\bm{x}_k$ comprises
the $n$ model parameters described in Section \ref{phm} at time step $k$. Note
that $\bm{x}_k$ also includes $T$, $M$, and $X$. We distinguish here between the
state of the lake and observations of this state ($T$, $M$, $X$) which are to
some degree uncertain. Equation \ref{bsm} can be seen as the probabilistic
formulation of the inverse problem inferring $\dot{Q_i}$ from available
observations.  

Solving Equation \ref{bsm} using standard strategies like Markov Chain Monte
Carlo is computationally very expensive as the number of variables grows
exponentially with the length of the timeseries and the number of model
parameters. Alternatives are Bayesian Filtering and Smoothing algorithms of
which the Kalman Filter and Smoother are the closed form solution for linear
models \cite{Kalman1960, Rauch1965, sarkkaBayesianFilteringSmoothing2013}.

Equation \ref{bsm} is the conditional probability of $\bm{x}_k$ given all
observations until time $t$. We assume that the states of the lake form a
Markov chain, meaning that the state of the lake at time step $k$, $\bm{x}_k$,
only depends on the state at adjacent time steps:

\begin{equation}\label{markov1}
    p(\bm{x}_k|\bm{x}_{1:k-1}, \bm{y}_{1:k-1}) = p(\bm{x}_k|\bm{x}_{k-1})
\end{equation}

and

\begin{equation}\label{markov2}
    p(\bm{x}_{k-1}|\bm{x}_{k:t}, \bm{y}_{k:t}) = p(\bm{x}_{k-1}|\bm{x}_{k})
\end{equation}

With this assumption and using \textit{Bayes' rule}, Equation \ref{bsm} can be
rewritten \cite{sarkkaBayesianFilteringSmoothing2013} as:

\begin{equation}\label{bsm1}
p(\bm{x}_k|\bm{y}_{1:t}) = p(\bm{x}_k|\bm{y}_{1:k})
    \int\left[\frac{p(\bm{x}_{k+1}|\bm{x}_k)p(\bm{x}_{k+1}|\bm{y}_{1:t})}{p(\bm{x}_{k+1}|\bm{y}_{1:k})}\right]d\bm{x}_{k+1}
\end{equation}

with

\begin{eqnarray}\label{bfilt1}
p(\bm{x}_{k+1}|\bm{y}_{1:k}) & = & \int p(\bm{x}_{k+1}|\bm{x}_k)p(\bm{x}_k|\bm{y}_{1:k})d\bm{x}_k \\
        \label{bfilt}
    p(\bm{x}_k|\bm{y}_{1:k}) & = & \frac{1}{z_k} p(\bm{y}_k|\bm{x}_k)p(\bm{x}_k|\bm{y}_{1:k-1}) \\
        \label{zetk}
    z_k & = & \int p(\bm{y}_k|\bm{x}_k) p(\bm{x}_k| \bm{y}_{1:k-1}) d\bm{x}_k  
\end{eqnarray}

using the \textit{Chapman-Kolmogorov} equation
\cite{sarkkaBayesianFilteringSmoothing2013} we can write:

\begin{equation}
p(\bm{x}_k|\bm{y}_{1:k-1}) = \int p(\bm{x}_k|\bm{x}_{k-1})p(\bm{x}_{k-1}|\bm{y}_{1:k-1})d\bm{x}_{k-1} 
\end{equation}

Equations \ref{bsm1} to \ref{zetk} describe a recursive way to compute Equation
\ref{bsm} in which we only need to know $\bm{x}_1$, $p(\bm{x}_k|\bm{x}_{k-1})$,
and $p(\bm{y}_k|\bm{x}_k)$. Equation \ref{bsm1} and \ref{bfilt} are also known
as the Bayesian Smoothing and Bayesian Filtering equation, respectively. If
$p(\bm{x}_k|\bm{x}_{k-1})$ and $p(\bm{y}_k|\bm{x}_k)$ are linear transformations
with normally distributed errors, the Kalman filter \cite{Kalman1960} and Kalman
smoother \cite{Rauch1965} represent closed form solutions. In our case, these
transformations are:

\begin{eqnarray}
p(\bm{x}_k|\bm{x}_{k-1}) & = & f(\bm{x}_{k-1}) + \bm{q}_{k-1} \\
    p(\bm{y}_k|\bm{x}_k) & = & \bm{x}_k[1:m] + \bm{r}_k   
\end{eqnarray}

where $f(\bm{x}_{k-1})$ is the non-linear physical model described in Section
\ref{phm} propagated forward in time using a fourth-order Runge-Kutta scheme. If
we assume the errors, $\bm{r}_k$ and $\bm{q}_{k}$, to be normally distributed
($\bm{q}_{k} \sim N(\bm{0}, \bm{Q}_{k})$, $\bm{r}_k \sim N(\bm{0}, \bm{R}_k)$),
we can compute an approximate solution to Equation \ref{bsm1} using the
Unscented Kalman Smoother \citep{Merwe2004,
sarkkaBayesianFilteringSmoothing2013}, a non-linear Kalman Smoother.

We estimate $\bm{R}_k$ from the daily variations in the observations but we do
not know $\bm{x}_1$ and $\bm{Q}_{k}$. Setting $\bm{x}_1$ to an arbitrary value
within physically reasonable bounds but large uncertainty works well in practice
and has very little influence on the results. While $\bm{Q}_{k}$ can be formally
optimised we treat it here as a design parameter that we modify such that the
observations are matched within their error bounds and the inferred heat input
rate varies smoothly. The latter is motivated by the fact that the physical
model described in \ref{phm} is not able to model short-term transient changes
in lake volume, temperature, and ion concentration. These can be caused by rain
or increased meltwater influx, or the influx of ion-enriched fluids. The
underlying assumption of a perfectly mixed lake typically does not apply in
these situations. 

The ability to account for these epistemic uncertainties explicitly is a
particular strength of the Bayesian Filtering and Smoothing equations. It allows
us to gain valuable insights even from very simple models. 

\section{Results}
\subsection{Synthetic test}\label{syn_test} 

To evaluate the inference scheme we design a synthetic test, consisting of a
very large cylinder, the same volume as RCL. We keep the windspeed and surface
water inflow constant at 4.5 m/s and 10 kt/day, respectively. To model the
outflow we assume the outlet to be a pipe 0.2 $m^2$ in cross section and then
use Bernoulli's equation to model the water outflow based on the water level in
the cylinder above the pipe. We assume that the incoming heat follows a
combination of a third-order polynomial and a sinusoid function and that the
enthalpy of the incoming steam is 2.8 MJ/kg which corresponds to the enthalpy of
saturated steam at hydrostatic pressures similar to those at the bottom of the
lake \cite{Mayhew1978}. To the resulting synthetic observations we add
uncertainties similar to those estimated from real observations (see Section
\ref{realdata}). As with real data, average and standard deviation of all values
within a day are calculated. The standard deviation is taken as a proxy for the
uncertainty in observations ($\bm{R}_k$). To simulate the disparity between sampling
intervals of different data streams we randomly remove most of the observations
of $X$, $\dot{M_o}$, and $\dot{M_s}$. This dataset is then inverted using the
Unscented Kalman Smoother to recover the original heat input rate. 

Figure \ref{syn_example} shows the result of the probabilistic inference of heat
input compared to the true input of the synthetic model. The observations fit
well within their error bounds but we see some descrepancies between the
original and the recovered heat input rates, especially for very steep changes
in the input rate. This is the result of preferring a smooth solution as already
mentioned in Section \ref{inf}.

It is noteworthy that the mass outflux rates are recovered quite accurately
despite providing very few synthetic observations.

\subsection{Real data}\label{realdata} 

Temperature at RCL is measured by an integrated-circuit temperature device
(Texas Instruments LM35) located roughly 1.9 m below the lake's current overflow
level. The water level sensor is a hydrostatic pressure sensor (First Sensor
KTE/KTU/KTW6000\ldots CS Series) and it is co-located with the temperature
sensor. Both sensors have a sampling interval of 15 minutes and transmit data
several times a day via satellite to GNS Science, which monitors New Zealand's
geohazards through its GeoNet project \cite{GeoNetHome}. Due to the type of
satellite connection only about 30\% of the temperature and level measurements
arrive at GeoNet's datacenter; the rest of the data is lost. Water samples are
taken manually 8-12 times a year from which subsequently, among many other
analyses, concentrations of $Mg^{++}$ and $Cl^-$ are determined at GNS Science's
Geothermal Analytical Laboratory. We will focus here on $Mg^{++}$ concentration
as it shows less deviation from the assumption of constant dilution than $Cl^-$.
Because of the sparse sampling, new information on $Mg^{++}$ concentration is
available approximately once per month.
 
To estimate uncertainties of temperature and water level observations, we 
compute their daily averages and standard deviations. Uncertainties of $Mg^{++}$
concentrations are estimated directly when two samples were taken on the same day.
Otherwise, their uncertainty is taken to be the median of directly estimated
uncertainties from when multiple samples were taken. 

Lake outflow ($\dot{M_o}$) is measured once or twice per year and we further
know from field observations that the lake is not overflowing beneath a certain
water level. We fit a sigmoid function to these observations assuming a maximum
outflow rate of 250 l/s. We assume the uncertainty in $\dot{M_o}$ as the 95
percent confidence interval of the curve-fitting.

As there is no permanent weather station close to RCL we assume the wind speed
to be 5 m/s. This was the average windspeed derived from a temporary deployment
of a weather buoy on RCL and adjacent permanent weather stations
\cite{hurstUseWeatherBuoy2012}.

Figure \ref{inference_result}a shows $\dot{Q_i}$ for RCL for lake measurements
between 4 March 2016 and 1 February 2022. During this time the heat input rate
reached a maximum of 753 MW and was on average 165 MW. The average standard
deviation was 34 MW and generally increased for lower values of $\dot{Q_i}$. To
put this in context, currently installed geothermal power plants in New Zealand
have a combined output of $\sim$~1000MW.

To also investigate changes in periodicity we computed the continuous wavelet
transform of of $\dot{Q_i}$ using a complex Morlet wavelet (Figure
\ref{inference_result}b). We chose the wavelet transform over the more common
windowed Fourier transform for its ability to detect non-stationary signals
\cite{torrencePracticalGuideWavelet1998}. Figure \ref{inference_result}c shows
the global wavelet spectrum which is an average of the wavelet analysis in
Figure \ref{inference_result}b over the whole analysis period. The global
wavelet spectrum can be interpreted similarly to the Power Spectral Density.
Figure \ref{inference_result}c shows that periodicity is dominated by periods
between 100 and 250 days. Which period is dominant at any one time changes
significantly and we will expand on this further in the following section.  

\section{Discussion and conclusion}
Inferring the heat input rate ($\dot{Q_i}$) into a crater lake is an important
part of estimating the total energy budget of a volcanic system
\cite{Brown1989a}. Similar to a volcano's gas budget, changes in the heat input
rate can either indicate changes in the magmatic heat source or in the
hydrothermal system.

Experiments on analog models of hydrothermal systems suggest that
changes in the heat source and the hydrothermal system should also change the
length of heating cycles
\cite{vandemeulebrouckAnalogueModelingInstabilities2005,fitzgeraldStabilityTwoPhaseGeothermal1997}.
Figure \ref{inference_result}b shows the heating cycles before mid-2017 and
after mid-2019 were dominated by longer periods. From mid-2017 to mid-2019
cycles were mostly of shorter periods which may indicate a change in the
volcanic system during this time. 

This is further supported by the cumulative difference to the mean for the heat
input rate as shown in Figure \ref{cumulative}. This type of graph helps
highlighting periods when the heat input rate was above or below the long-term
mean. Before 2017 and after 2019 the difference appears to fluctuate more or
less around the mean value. After a dip beginning in 2017 a period of higher
than mean values starts towards the end of 2017 and ends early 2020. 

To further explore hypotheses like this, the next logical step is to include
numerical models of magmatic gas and heat release as well as hydrothermal heat
and fluid flow. Our inversion method is well suited for such more complex models
as its runtime scales linearly with the number of parameters. Several studies
have demonstrated the value of non-linear Kalman Filters for inverse problems
with non-linear, high-dimensional numerical models
\cite[e.g.][]{White2018a,huangIteratedKalmanMethodology2022}. 

Previous studies of RCL's mass and energy balance found that they required very
high values of enthalpy for the incoming steam to avoid violating physical
boundary conditions such as negative values for the inflow rate of meteoric
water ($\dot{M_s}$) \cite{Hurst1991, hurstCraterLakeEnergy2015}. Their
explanation was that as rising hot fluids enter the lake, relatively cool lake
water flows back into the vent. This would lead to an overall lower addition of
mass to the lake and resulting in an enthalpy value that is double the expected
for saturated steam at hydrostatic pressures likely to be found at the bottom of
the lake. By including uncertainties in our observations, this explanation is
still a possibility but no longer a requirement to satisfy the physical
constraints. Numerical modelling of hydrothermal processes may also shed further
light on the heating processes and whether or not the lake and the vent form a
single or two separate convective systems.

In conclusion, the combination of the mass and energy model, developed in
previous studies \cite{Hurst1991,hurstCraterLakeEnergy2015}, with the
probabilistic inference method we developed here provides a powerful method to
continuously infer the heat input rate into RCL. It allows us to combine
disparate data streams, invert them for probabilistic estimates of the heat
input rate and thereby keep a finger on Mt. Ruapehu's energy budget.

\appendix
\section{Surface losses}\label{A}
\subsection{long-wavelength radiation}
The net loss through long-wavelength radiation can be written as:
\begin{equation}
    Q_{rad} = A\sigma(\epsilon_w T_s^4 - \epsilon_a T_a^4)
\end{equation}

where
\begin{table}[h!]
\begin{tabular}{cp{8cm}}
    A & is the surface of the lake (in $m^2$) \\
    $\sigma$ & is the Stefan-Boltzmann coefficient ($5.67e^{-8} \frac{W}{m^2 K^4}$) \\
    $\epsilon_w$ & is the lake emissivity \\
    $T_s$ & is the water surface temperature (here assumed to be 1 $^{\circ}C$ less than the
    measured temperature) \\
    $\epsilon_a$ & is the effective atmospheric emissivity \\
    $T_a$ & is the effective air temperature (here assumed to be 0.9 $^{\circ}C$) \\ 
\end{tabular}
\end{table}

\subsection{Evaporation}
As proposed by \cite{hurstCraterLakeEnergy2015} we use the equation of
\cite{adamsEvaporationHeatedWater1990} but replace the term for forced
convection by that of \cite{sartoriCriticalReviewEquations2000}:

\begin{equation}
    Q_{evap} = [(2.2(T_{sv} - T_{av})^{1/3}(e_s - e_a))^2 + 
    (L(0.00407u^{0.8}X_0^{-0.2})(e_s - e_a)/P_a)^2]^{1/2}
\end{equation}

where
\begin{table}[h!]
\begin{tabular}{cp{8cm}}
    $T_{sv}$, $T_{av}$ & are the virtual surface and air temperatures respectively.
    This correction accounts for the extra buoyancy of water vapour compared to air
    (see Equation \ref{vtemp} for details) \\
    $e_s$, $e_a$ & are the saturation vapour pressure and the ambient air vapour
    pressure respectively \\
    $L$ & is the latent heat of evaporation \\
    $u$ & is the wind speed \\
    $X_0$ & is the fetch, or characteristic length, of the lake\\
    $P_a$ & is the atmospheric pressure \\
\end{tabular}
\end{table}

\begin{equation}\label{vtemp}
T_{xv} = \frac{T_x}{(1-0.378e_x/P_a)} \qquad \mathrm{for}\quad x = s\ \mathrm{or}\ a
\end{equation}

\subsection{Sensible heat}
This describes the heat loss due to conduction and convection above a hot lake
and according to \cite{Stevenson1992} the ratio between sensible heat,
$Q_{sh}$, and heat loss due to evaporation, $Q_{evap}$ can be written as:

\begin{equation}
    \frac{Q_{sh}}{Q_{evap}}=R_{sh}=\frac{\rho c_a}{L} \frac{(T_s - T_a)}{(e_s - e_a)}
    \underbrace{\left [ \frac{(T_s-T_a)}{(T_{sv} - T_{av})} \right ]^{1/3}}_A 
\end{equation}

where

\begin{table}[h!]
\begin{tabular}{cp{8cm}}
    $\rho$ & is the air density \\
    $c_a$ & is the specific heat of air\\
\end{tabular}
\end{table}


We decided to ignore term A as it tends to be very close to 1.

\subsection{Putting it all together}
The final equation to compute surface heat losses is then:
\begin{equation}
    Q_e = Q_{rad} + Q_{evap}(1+R_{sh})
\end{equation}

and the mass-loss due to evaporation is:
\begin{equation}
    M_e = Q_{evap}/L
\end{equation}




%\nocite{oreg,schn,pond,smith,marg,hunn,advi,koha,mouse}

%%%%%%%%%%%%%%%%%%%%%%%%%%%%%%%%%%%%%%%%%%%%%%
%%                                          %%
%% Backmatter begins here                   %%
%%                                          %%
%%%%%%%%%%%%%%%%%%%%%%%%%%%%%%%%%%%%%%%%%%%%%%

\begin{backmatter}

\section*{Declarations}
\subsection*{Acknowledgements}
The authors would like to thank Nico Fournier for providing feedback and ideas
throughout the project and Sophie Pearson-Grant for her constructive and
insightful feedback on this manuscript. GNS Science's volcano monitoring group
has been an invaluable team to test and improve this work. Graphs have been
produced using the open-source graphing libraries Plotly
(https://plotly.com/python/), matplotlib (https://matplotlib.org/) and cartopy
(https://scitools.org.uk/cartopy/docs/latest/).

\subsection*{Funding}%% if any
This project was funded in parts by the Strategic Science Investment Funding to
GNS Science within the Datascience programme and by the GeoNet programme.

\subsection*{Abbreviations}%% if any
  \begin{tabular}{ll}
    RCL  & Ruapehu Crater Lake \\ 
    $\dot{M}$ & Lake water mass change \\
    $\dot{M_i}$ & Steam input \\ 
    $\dot{M_s}$ & Melt water and precipitation inflow \\
    $\dot{M_o}$ & Seepage and overflow \\ 
    $H$ & Enthalpy \\
    $T$ & Lake temperature \\
    $c_w$ & Specific heat of water \\
    $\dot{Q_i}$ & Energy input from volcanic sources \\
    $\dot{Q_r}$ & Solar irradiation \\
    $\dot{Q_e}$ & Surface energy loss \\
    $X$ & Total ion amount in the lake\\
  \end{tabular}

\subsection*{Availability of data and materials}%% if any
The source code for this work can be accessed at
https://github.com/yannikbehr/pumahu. All data used in this study is provided
free of charge by the GeoNet program (https://www.geonet.org.nz).

\subsection*{Ethics approval and consent to participate}%% if any
Not applicable

\subsection*{Competing interests}
The authors declare that they have no competing interests.

\subsection*{Consent for publication}%% if any
Not applicable

\subsection*{Author's contributions}
  Y.B. wrote the main manuscript text, produced graphs, and implemented the
  Kalman Smoothing. S.S. worked on data collection and interpretation. T.H.
  developed and implemented the mass and energy balance model. All
  authors reviewed the manuscript.

\subsection*{Authors' information}%% if any
All authors are affiliated with GNS Science. TH is Emeritus Volcano
Geophysicist, SS is Senior Volcano Geophysicist and YB is Volcano Science
Operations Specialist.


%%%%%%%%%%%%%%%%%%%%%%%%%%%%%%%%%%%%%%%%%%%%%%%%%%%%%%%%%%%%%
%%                  The Bibliography                       %%
%%                                                         %%
%%  Bmc_mathpys.bst  will be used to                       %%
%%  create a .BBL file for submission.                     %%
%%  After submission of the .TEX file,                     %%
%%  you will be prompted to submit your .BBL file.         %%
%%                                                         %%
%%                                                         %%
%%  Note that the displayed Bibliography will not          %%
%%  necessarily be rendered by Latex exactly as specified  %%
%%  in the online Instructions for Authors.                %%
%%                                                         %%
%%%%%%%%%%%%%%%%%%%%%%%%%%%%%%%%%%%%%%%%%%%%%%%%%%%%%%%%%%%%%

% if your bibliography is in bibtex format, use those commands:
\bibliographystyle{bmc-mathphys} % Style BST file (bmc-mathphys, vancouver, spbasic).
\bibliography{zotero}      % Bibliography file (usually '*.bib' )
% for author-year bibliography (bmc-mathphys or spbasic)
% a) write to bib file (bmc-mathphys only)
% @settings{label, options="nameyear"}
% b) uncomment next line
%\nocite{label}

% or include bibliography directly:
% \begin{thebibliography}
% \bibitem{b1}
% \end{thebibliography}

%%%%%%%%%%%%%%%%%%%%%%%%%%%%%%%%%%%
%%                               %%
%% Figures                       %%
%%                               %%
%% NB: this is for captions and  %%
%% Titles. All graphics must be  %%
%% submitted separately and NOT  %%
%% included in the Tex document  %%
%%                               %%
%%%%%%%%%%%%%%%%%%%%%%%%%%%%%%%%%%%

%%
%% Do not use \listoffigures as most will included as separate files
%\newpage
\section*{Figures}

\begin{figure}[h!]
    \includegraphics[width=12 cm]{figures/overview.png}  
    \caption{Image of Mt. Ruapehu's crater lake from 2008 (\textcopyright Visual
        Media Library, GNS Science, ID 6277, photographer Karen Britten). The red
        circle marks the location of the lake outlet. The location of Mt. Ruapehu is
        shown as a red triangle on the inset map.}
      \label{overview}
\end{figure}

\begin{figure}[h!]
   \includegraphics[width=12 cm]{figures/sensitivity_analysis.png}  
    \caption{Surface energy loss ($Q_e$) with respect to lake water temperature
             ($T$). The model parameters that have been held fixed are shown in  
             the title.}
      \label{sens}
\end{figure}
    
\begin{figure}[h!]
	\includegraphics[width=12 cm]{figures/synthetic_inversion_uks.png}  
    \caption{Inversion result for the synthetic test described in Section
        \ref{syn_test}. Turquoise lines and markers represent synthetic
        observations; red dashed lines are the inversion results and black
        dashed lines are the true input to the synthetic test. Error bars and
        shaded areas represent $\pm$ 3 standard deviations.}
  \label{syn_example}
\end{figure}

\begin{figure}[h!]
 	\includegraphics[width=12 cm]{figures/time_frequency_analysis.png}  
    \caption{ a) Inference results for the heat input rate $\dot{Q_i}$ at RCL
             between 4 March 2016 and 1 February 2022. The solid line shows the
             marginal probability and the uncertainty is displayed as shaded
             region; b) Continuous wavelet transform of the time series shown in
             a; c) Power spectral density of the heat input rate shown as a
             solid line with uncertainty displayed as shaded region.}
\label{inference_result}
\end{figure}

\begin{figure}[h!]
 	\includegraphics[width=12 cm]{figures/cumulative_heat.png}  
    \caption{Cumulative difference to the mean for the heat input rate into RCL
             (solid line) and its uncertainty (shaded region) calculated from
             the time-series in Figure \ref{inference_result}a.} 
\label{cumulative}
\end{figure}



%%   \begin{figure}[h!]
%%   \caption{\csentence{Sample figure title.}
%%       A short description of the figure content
%%       should go here.}
%%       \end{figure}
%% 
%% \begin{figure}[h!]
%%   \caption{\csentence{Sample figure title.}
%%       Figure legend text.}
%%       \end{figure}

%%%%%%%%%%%%%%%%%%%%%%%%%%%%%%%%%%%
%%                               %%
%% Tables                        %%
%%                               %%
%%%%%%%%%%%%%%%%%%%%%%%%%%%%%%%%%%%

%% Use of \listoftables is discouraged.
%%
%% \section*{Tables}
%% \begin{table}[h!]
%% \caption{Sample table title. This is where the description of the table should go.}
%%       \begin{tabular}{cccc}
%%         \hline
%%            & B1  &B2   & B3\\ \hline
%%         A1 & 0.1 & 0.2 & 0.3\\
%%         A2 & ... & ..  & .\\
%%         A3 & ..  & .   & .\\ \hline
%%       \end{tabular}
%% \end{table}

%%%%%%%%%%%%%%%%%%%%%%%%%%%%%%%%%%%
%%                               %%
%% Additional Files              %%
%%                               %%
%%%%%%%%%%%%%%%%%%%%%%%%%%%%%%%%%%%

%% \section*{Additional Files}
%%   \subsection*{Additional file 1 --- Sample additional file title}
%%     Additional file descriptions text (including details of how to
%%     view the file, if it is in a non-standard format or the file extension).  This might
%%     refer to a multi-page table or a figure.
%% 
%%   \subsection*{Additional file 2 --- Sample additional file title}
%%     Additional file descriptions text.


\end{backmatter}
\end{document}
