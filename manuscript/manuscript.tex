%% BioMed_Central_Tex_Template_v1.06
%%                                      %
%  bmc_article.tex            ver: 1.06 %
%                                       %

%%IMPORTANT: do not delete the first line of this template
%%It must be present to enable the BMC Submission system to
%%recognise this template!!

%%%%%%%%%%%%%%%%%%%%%%%%%%%%%%%%%%%%%%%%%
%%                                     %%
%%  LaTeX template for BioMed Central  %%
%%     journal article submissions     %%
%%                                     %%
%%          <8 June 2012>              %%
%%                                     %%
%%                                     %%
%%%%%%%%%%%%%%%%%%%%%%%%%%%%%%%%%%%%%%%%%


%%%%%%%%%%%%%%%%%%%%%%%%%%%%%%%%%%%%%%%%%%%%%%%%%%%%%%%%%%%%%%%%%%%%%
%%                                                                 %%
%% For instructions on how to fill out this Tex template           %%
%% document please refer to Readme.html and the instructions for   %%
%% authors page on the biomed central website                      %%
%% http://www.biomedcentral.com/info/authors/                      %%
%%                                                                 %%
%% Please do not use \input{...} to include other tex files.       %%
%% Submit your LaTeX manuscript as one .tex document.              %%
%%                                                                 %%
%% All additional figures and files should be attached             %%
%% separately and not embedded in the \TeX\ document itself.       %%
%%                                                                 %%
%% BioMed Central currently use the MikTex distribution of         %%
%% TeX for Windows) of TeX and LaTeX.  This is available from      %%
%% http://www.miktex.org                                           %%
%%                                                                 %%
%%%%%%%%%%%%%%%%%%%%%%%%%%%%%%%%%%%%%%%%%%%%%%%%%%%%%%%%%%%%%%%%%%%%%

%%% additional documentclass options:
%  [doublespacing]
%  [linenumbers]   - put the line numbers on margins

%%% loading packages, author definitions

%\documentclass[twocolumn]{bmcart}% uncomment this for twocolumn layout and comment line below
\documentclass{bmc_template/bmcart}

%%% Load packages
%\usepackage{amsthm,amsmath}
\RequirePackage{natbib}
%\RequirePackage[authoryear]{natbib}% uncomment this for author-year bibliography
%\RequirePackage{hyperref}
\usepackage[utf8]{inputenc} %unicode support
%\usepackage[applemac]{inputenc} %applemac support if unicode package fails
%\usepackage[latin1]{inputenc} %UNIX support if unicode package fails
\usepackage{graphicx}


%%%%%%%%%%%%%%%%%%%%%%%%%%%%%%%%%%%%%%%%%%%%%%%%%
%%                                             %%
%%  If you wish to display your graphics for   %%
%%  your own use using includegraphic or       %%
%%  includegraphics, then comment out the      %%
%%  following two lines of code.               %%
%%  NB: These line *must* be included when     %%
%%  submitting to BMC.                         %%
%%  All figure files must be submitted as      %%
%%  separate graphics through the BMC          %%
%%  submission process, not included in the    %%
%%  submitted article.                         %%
%%                                             %%
%%%%%%%%%%%%%%%%%%%%%%%%%%%%%%%%%%%%%%%%%%%%%%%%%


% \def\includegraphic{}
% \def\includegraphics{}



%%% Put your definitions there:
\startlocaldefs
\newcommand{\bm}[1]{{\bf #1}}
\endlocaldefs


%%% Begin ...
\begin{document}

%%% Start of article front matter
\begin{frontmatter}

\begin{fmbox}
\dochead{Research}

%%%%%%%%%%%%%%%%%%%%%%%%%%%%%%%%%%%%%%%%%%%%%%
%%                                          %%
%% Enter the title of your article here     %%
%%                                          %%
%%%%%%%%%%%%%%%%%%%%%%%%%%%%%%%%%%%%%%%%%%%%%%

\title{Continuous estimates of heat emission at Mt. Ruapehu using
the Unscented Kalman Smoother}

%
%%%%%%%%%%%%%%%%%%%%%%%%%%%%%%%%%%%%%%%%%%%%%%
%%                                          %%
%% Enter the authors here                   %%
%%                                          %%
%% Specify information, if available,       %%
%% in the form:                             %%
%%   <key>={<id1>,<id2>}                    %%
%%   <key>=                                 %%
%% Comment or delete the keys which are     %%
%% not used. Repeat \author command as much %%
%% as required.                             %%
%%                                          %%
%%%%%%%%%%%%%%%%%%%%%%%%%%%%%%%%%%%%%%%%%%%%%%

\author[
   addressref={aff1},                   % id's of addresses, e.g. {aff1,aff2}
   corref={aff1},                       % id of corresponding address, if any
   noteref={n1},                        % id's of article notes, if any
   email={y.behr@gns.cri.nz}   % email address
]{\inits{YB}\fnm{Yannik} \snm{Behr}}
\author[
   addressref={aff1},
   email={s.sherburn@gns.cri.nz}
]{\inits{SS}\fnm{Steve} \snm{Sherburn}}
\author[
   addressref={aff1},
   email={t.hurst@gns.cri.nz}
]{\inits{TH}\fnm{Tony} \snm{Hurst}}


%%%%%%%%%%%%%%%%%%%%%%%%%%%%%%%%%%%%%%%%%%%%%%
%%                                          %%
%% Enter the authors' addresses here        %%
%%                                          %%
%% Repeat \address commands as much as      %%
%% required.                                %%
%%                                          %%
%%%%%%%%%%%%%%%%%%%%%%%%%%%%%%%%%%%%%%%%%%%%%%

\address[id=aff1]{%
    \orgname{GNS Science},
  \street{114 Karetoto Rd},
  \postcode{3384}
  \city{Taupo},
  \cny{New Zealand}
}

%%%%%%%%%%%%%%%%%%%%%%%%%%%%%%%%%%%%%%%%%%%%%%
%%                                          %%
%% Enter short notes here                   %%
%%                                          %%
%% Short notes will be after addresses      %%
%% on first page.                           %%
%%                                          %%
%%%%%%%%%%%%%%%%%%%%%%%%%%%%%%%%%%%%%%%%%%%%%%

\begin{artnotes}
%\note{Sample of title note}     % note to the article
%\note[id=n1]{Equal contributor} % note, connected to author
\end{artnotes}

\end{fmbox}% comment this for two column layout

%%%%%%%%%%%%%%%%%%%%%%%%%%%%%%%%%%%%%%%%%%%%%%
%%                                          %%
%% The Abstract begins here                 %%
%%                                          %%
%% Please refer to the Instructions for     %%
%% authors on http://www.biomedcentral.com  %%
%% and include the section headings         %%
%% accordingly for your article type.       %%
%%                                          %%
%%%%%%%%%%%%%%%%%%%%%%%%%%%%%%%%%%%%%%%%%%%%%%

\begin{abstractbox}

\begin{abstract} % abstract
\parttitle{First part title} %if any
Text for this section.

\parttitle{Second part title} %if any
Text for this section.
\end{abstract}

%%%%%%%%%%%%%%%%%%%%%%%%%%%%%%%%%%%%%%%%%%%%%%
%%                                          %%
%% The keywords begin here                  %%
%%                                          %%
%% Put each keyword in separate \kwd{}.     %%
%%                                          %%
%%%%%%%%%%%%%%%%%%%%%%%%%%%%%%%%%%%%%%%%%%%%%%

\begin{keyword}
\kwd{sample}
\kwd{article}
\kwd{author}
\end{keyword}

% MSC classifications codes, if any
%\begin{keyword}[class=AMS]
%\kwd[Primary ]{}
%\kwd{}
%\kwd[; secondary ]{}
%\end{keyword}

\end{abstractbox}
%
%\end{fmbox}% uncomment this for twcolumn layout

\end{frontmatter}

%%%%%%%%%%%%%%%%%%%%%%%%%%%%%%%%%%%%%%%%%%%%%%
%%                                          %%
%% The Main Body begins here                %%
%%                                          %%
%% Please refer to the instructions for     %%
%% authors on:                              %%
%% http://www.biomedcentral.com/info/authors%%
%% and include the section headings         %%
%% accordingly for your article type.       %%
%%                                          %%
%% See the Results and Discussion section   %%
%% for details on how to create sub-sections%%
%%                                          %%
%% use \cite{...} to cite references        %%
%%  \cite{koon} and                         %%
%%  \cite{oreg,khar,zvai,xjon,schn,pond}    %%
%%  \nocite{smith,marg,hunn,advi,koha,mouse}%%
%%                                          %%
%%%%%%%%%%%%%%%%%%%%%%%%%%%%%%%%%%%%%%%%%%%%%%

%%%%%%%%%%%%%%%%%%%%%%%%% start of article main body
% <put your article body there>

%%%%%%%%%%%%%%%%
%% Background %%
%%
\section*{Introduction}

The amount and rate of heat emitted by a volcano provides important clues about
volcanic processes at depth. For example, a sudden rise in the heat output rate
combined with increased microgravity signals may result from the emplacement of
fresh magma at shallow depth \cite{Brown1991}. A slow decrease in the heat
emission rate together with a high carbon-sulfur ratio of the emitted gas may
point to the formation of a hydrothermal seal.

Volcanic lakes help with estimating heat emissions. Often, the majority of a
volcano's heat and gas output passes through the crater lake. Crater lakes act thus
like a filter, integrating heat and gas input from the vents entering the lake.

Monitoring changes in temperature, water mass, and ion concentration in these
lakes can thus serve as proxies for changes in gas- and steam-release by the
hydrothermal system and, ultimately, the melt zone beneath the volcano. 

Crater Lake (Te Wai A-Moe) on Mt. Ruapehu, an active andesitic stratovolcano in
the center of New Zealand's North Island (Figure \ref{overview}), is one such
example with records of observations dating back to the 19th century
\citep[e.g.][]{Friedlander1898}. Mt. Ruapehu erupted at least 602 times since
1830 with many phreatic and phreato-magmatic eruptions and two major magmatic
episodes \citep{Scott2013}. Although the latest magmatic event between June
1995 to November 1997 and a subsequent dam collapse changed the shape and
volume of the lake, it has been largely unchanged since then and the lake's
bathymetry is well known from times when it was empty [Tony Hurst, personal
communication].

Increases in heat and steam input may be caused by the ascent of magma or
changes in the permeability of the hydrothermal system and are, therefore,
important parameters to be monitored. First attempts to quantify the amount of
heat and steam entering the lake date back to \citep{Dibble1966}. Over the past
decade water temperature and lake level have been monitored continuously
and water samples for ion concentration measurements have been taken roughly
every two to three months. This enables us to continuously estimate heat and
steam input into the lake and, in principle, automatically detect changes in
these variables as soon as they occur.

Inference of the heat and steam input is typically done through a mass and
energy balance calculation: the sum of mass and energy flow into and out of the
lake has to account for the observed changes in lake temperature, water level
and ion dilution \citep[e.g.,][]{Hurst1981, Hurst1991, Stevenson1992,
Fournier2009, Scott1994}. At Crater Lake, and probably at most other volcanic
lakes, this is an underdetermined problem, that is the number of unknowns
exceeds the number of constraints. Hence, it is necessary to constrain some of
the parameters in the mass and energy balance model using prior knowledge. Further,
some observations and input parameters come with large uncertainties and are sampled
irregularly.

We propose a probabilistic (Bayesian) inference approach which naturally
handles these difficulties and transparently propagates uncertainties of
observations and prior assumptions into posterior uncertainties of heat and
steam input. The accurate description of uncertainties is of particular
importance when publishing automatically generated estimates of heat and steam
input, as they could potentially be used in decision making or automatic
alerting of monitoring staff.

\begin{figure}
\includegraphics[width=\textwidth]{figures/overview.png}  
\caption{Image of Mt. Ruapehu's crater lake from 2008 (\textcopyright Karen
	Britten, GNS Science). The red O marks the location of the lake outlet. The
	location of Mt. Ruapehu is shown as a red triangle on the insert map.}
  \label{overview}
\end{figure}


\section*{Probabilistic inference method}\label{Pim}

\subsection*{Physical model}\label{phm}

Following \citet{Hurst1991} and \citet{Stevenson1992} the mass and energy balance
can be written in terms of three coupled ordinary differential equations. The
rate of lake water mass change can be described as:
\begin{equation}\label{ode_M}
	\frac{dM}{dt} = M_s - M_o - M_e + M_i
\end{equation}

Where $M_s$ is the rate of water inflow through melt and precipitation, $M_o$
is the outflow rate through seepage and overflow, $M_e$ is the rate of mass
loss through evaporation and $M_i$ is the rate of steam input through the vent
system beneath the lake.

The thermal energy content of a mass flow can be described with the notion of
enthalpy, $H$, which is the ratio of energy and mass. Enthalpy can also be defined
on a relative scale as:
\begin{equation}
	H(T) = H(T_0) + \int_{T_0}^{T}c_p\tau d\tau
\end{equation}

where $T$ is temperature and $c_p$ is the specific heat. For all following
equations we take $H(T_0=0)$ as our reference enthalpy and $c_p=c_w=constant$,
with $c_w$ the specific heat of water, such that the enthalpy of lake water at
temperature $T$ can be written as $H(T) = c_w T$. Because Ruapehu Crater Lake
is at an altitude of ~2500 m and big parts of its catchment are covered by snow
and ice all year round we assume the temperature of incoming water to be at 0
$^{\circ}C$.

The rate of energy change of the lake water can thus be written as:
\begin{equation}\label{ode_E}
    \frac{dE}{dt}=\frac{d(c_wTM)}{dt}=c_wT\frac{dM}{dt} + c_wM\frac{dT}{dt} =
	-Q_e - c_wTM_o + Q_{eff} + Q_r
\end{equation}
      
where $T$ is the lake water temperature, $M$ is the lake water mass, $Q_r$ is
the rate of energy gain due to solar irradiation (short-wavelength radiation),
and $c_wTM_o$ is the rate of energy loss through outflow/seepage. $Q_e$ is
the rate of surface energy losses due to evaporation (forced and free
convection), sensible heat, and long-wavelength radiation (net loss). Several
empirical and theoretical equations exist to describe the surface losses and
both \citet{Stevenson1992} and \citet{Hurst2015} discuss them in detail. We
follow here mostly the choice of \citet{Hurst2015} as it includes the recent
development by \citet{Sartori2000} on forced (i.e. wind-induced) convection
(see Appendix \ref{A} for more details). $Q_{eff}$ is the effective rate of
energy gain from volcanic sources after heating incoming water:
\begin{equation}
	Q_{eff}=Q_i - M_s c_w T
\end{equation}
with $Q_i$ the rate of energy input from volcanic sources.

Following from Equation \ref{ode_E} the temperature change $\frac{dT}{dt}$ can then be written as:
\begin{equation}\label{ode_T}
	\frac{dT}{dt}=\frac{1}{c_wM}\left(-Q_e - c_wTM_o + Q_{eff} + Q_r\right )
	-\frac{T}{M}\frac{dM}{dt}
\end{equation}

Finally, the change in ion concentration of, for example, $Mg^{2+}$ can be expressed as:
\begin{equation}\label{ode_X}
	\frac{dX}{dt}=-M_o\frac{X}{M}
\end{equation}

Where $X$ is the total ion amount in the lake. This equation is only true under
the assumption that $X$ only changes due to dilution and there is no re-supply
between eruptive episodes. Equation \ref{ode_X} gives an estimate of the average
outflow over periods on the order of months but does not capture shorter term
variations in the outflow. 

\subsection*{Inference}\label{inf}

The objective of the inference problem is to estimate the amount and energy
content of steam and gas entering the lake from the hydrothermal system below.
To constrain these we use the currently available observations of lake
temperature ($T$), lake level ($L$), ion concentration ($X$) as well as
numerical weather model data on windspeed. Putting it more concisely, we are
looking for 

\begin{equation}\label{bsm}
    p(\bm{x}_k|\bm{y}_{1:T})
\end{equation}

where $\bm{y}_{1:T}$ is the time series of $m$ observations and $x_k$ comprises
the $n$ model parameters described in Section \ref{phm} at time step $k$. Note
that $x_k$ also includes the lake's temperature, lake level, and ion
concentration. We distinguish here between the state of the lake and
observations of this state ($T$, $L$, $X$) which are to some degree uncertain.

The following equations are based on \citet{Sarkka2010} which we refer the
reader to for more proofs and more detailed explanations.

Equation \ref{bsm} is the marginal probability of $\bm{x}_k$ given all
observations until time $T$. We assume that the states of the lake form a
Markov chain, meaning that the state of the lake at time step $k$, $\bm{x}_k$,
only depends on the state at adjacent time steps:

\begin{equation}\label{markov1}
    p(\bm{x}_k|\bm{x}_{1:k-1}, \bm{y}_{1:k-1}) = p(\bm{x}_k|\bm{x}_{k-1})
\end{equation}

and

\begin{equation}\label{markov2}
    p(\bm{x}_{k-1}|\bm{x}_{k:T}, \bm{y}_{k:T}) = p(\bm{x}_{k-1}|\bm{x}_{k})
\end{equation}

With this assumption and using \textit{Bayes' rule}, Equation \ref{bsm} can be
rewritten as:

\begin{equation}\label{bsm1}
p(\bm{x}_k|\bm{y}_{1:T}) = p(\bm{x}_k|\bm{y}_{1:k})
    \int\left[\frac{p(\bm{x}_{k+1}|\bm{x}_k)p(\bm{x}_{k+1}|\bm{y}_{1:T})}{p(\bm{x}_{k+1}|\bm{y}_{1:k})}\right]d\bm{x}_{k+1}
\end{equation}

with

\begin{eqnarray}\label{bfilt1}
p(\bm{x}_{k+1}|\bm{y}_{1:k}) & = & \int p(\bm{x}_{k+1}|\bm{x}_k)p(\bm{x}_k|\bm{y}_{1:k})d\bm{x}_k \\
        \label{bfilt}
    p(\bm{x}_k|\bm{y}_{1:k}) & = & \frac{1}{z_k} p(\bm{y}_k|\bm{x}_k)p(\bm{x}_k|\bm{y}_{1:k-1}) \\
        \label{zetk}
    z_k & = & \int p(\bm{y}_k|\bm{x}_k) p(\bm{x}_k| \bm{y}_{1:k-1}) d\bm{x}_k  
\end{eqnarray}

using the \textit{Chapman-Kolmogorov} equation we can write:

\begin{equation}
p(\bm{x}_k|\bm{y}_{1:k-1}) = \int p(\bm{x}_k|\bm{x}_{k-1})p(\bm{x}_{k-1}|\bm{y}_{1:k-1})d\bm{x}_{k-1} 
\end{equation}

Equations \ref{bsm1} to \ref{zetk} describe a recursive way to compute Equation
\ref{bsm} in which we only need to know $\bm{x}_1$,
$p(\bm{x}_k|\bm{x}_{k-1})$, and $p(\bm{y}_k|\bm{x}_k)$. Equation \ref{bsm} and
\ref{bfilt} are also known as the Bayesian Smoothing and Bayesian Filtering
equation, respectively. If $p(\bm{x}_k|\bm{x}_{k-1})$ and
$p(\bm{y}_k|\bm{x}_k)$ are linear transformations with normally distributed
errors, the \textit{Kalman filter} \citep{Kalman1960} and Kalman smoother
\citep{Rauch1965} represent closed form solutions. In our case, these
transformations are:

\begin{eqnarray}
p(\bm{x}_k|\bm{x}_{k-1}) & = & f(\bm{x}_{k-1}) + \bm{q}_{k-1} \\
    p(\bm{y}_k|\bm{x}_k) & = & \bm{x}_k[1:m] + \bm{r}_k   
\end{eqnarray}

where $f(\bm{x}_{k-1})$ is the physical model described in Section \ref{phm}
which is highly non-linear. If we assume the errors, $\bm{r}_k$ and
$\bm{q}_{k-1}$, to be normally distributed ($\bm{q}_{k-1} \sim N(\bm{0},
\bm{Q}_{k-1})$, $\bm{r}_k \sim N(\bm{0}, \bm{R}_k)$), we can compute an
approximate solution to \ref{bsm} using the \textit{Unscented Kalman Filter}
\citep{Merwe2004}.

We estimate $\bm{R}_k$ from the daily variations in the observations but we do
not know $\bm{x}_1$ and $\bm{Q}_{k-1}$. If $\bm{\theta}$ stands for all unknown
parameters than we are looking for $p(\bm{\theta}|\bm{y}_{1:T})$. Using again
\textit{Bayes' Theorem} and ignoring the normalising constant we can write:

\begin{equation}\label{pe1}
    p(\bm{\theta}|\bm{y}_{1:T}) \propto p(\bm{y}_{1:T}|\bm{\theta})p(\bm{\theta})
\end{equation}

Assuming that $\bm{Y}_{1:T}$ form a Markov chain and that $p(\bm{\theta})$ follows
a uniform distribution we can rewrite Equation \ref{pe1} as:

\begin{equation}\label{pe2}
    p(\bm{\theta}|\bm{y}_{1:T}) \propto \prod_{k=1}^Tp(\bm{y}_{k}|\bm{y}_{k-1}, \bm{\theta})
\end{equation}

Each factor on the right-hand side of Equation \ref{pe2} can be written as:
\begin{equation}
    p(\bm{y}_k|\bm{y}_{1:k-1}, \bm{\theta}) = \int p(\bm{y}_k|\bm{x}_k, \bm{\theta})p(\bm{x}_k|\bm{y}_{1:k-1})d\bm{x}_k
\end{equation}

which is readily computed as a by-product of Equation \ref{bfilt}. We then run
a non-linear optimisation to find the \textit{maximum a posteriori (MAP)}
estimate of Equation \ref{pe2}. 

* sensitivity analysis: high vs. low lake temperature; windspeed dependence;
  enthalpy dependence

\section*{Results}
\subsection*{Synthetic test}\label{syn_test}
To test the inversion scheme we design a synthetic test, consisting of a very
large cylinder, the same volume as Ruapehu Crater Lake. We keep the windspeed
and surface water inflow constant at 4.5 m/s and 10 kt/day, respectively. To
model the outflow we assume the outlet to be a pipe 0.2 $m^2$ in cross section
and then use Bernoulli's equation to model the water outflow based on the water
level in the cylinder above the pipe. We assume that the incoming heat follows
a combination of a third-order polynomial and a sinusoid function and that the
enthalpy of the incoming steam is 2.8 which corresponds to the enthalpy of
saturated steam at hydrostatic pressures similar to those at the bottom of the
lake \ref{Mayhew1978}. To the resulting synthetic observations we add
uncertainties similar to those estimated from real observations (see Section
\ref{realdata}).

Figure \ref{syn_example} shows the result of the probabilistic inference of
heat input compared to the true input of the synthetic model. The observations
are well fit within their error bounds but we see some descrepancies between
the original and the recovered heat input rates, especially for very steep
changes in the input rate. This is the result of preferring a smooth solution
as already mentioned in \ref{inf}.

It is noteworthy that the mass influx and outflux rates are recovered quite accurately despite having no data in the former and very little data in the latter case.

\begin{figure}
	\includegraphics[width=.9\textwidth]{figures/synthetic_inversion_uks.png}  
    \caption{Inversion result for the synthetic test described in Section 
        \ref{syn_test}. Turquoise lines and markers represent hypothetical
        observations; red dashed lines are the inversion results and black
        dashed lines are the true input to the synthetic test.}
  \label{syn_example}
\end{figure}


\section*{Section title}
Text for this section \ldots
\subsection*{Sub-heading for section}
Text for this sub-heading \ldots
\subsubsection*{Sub-sub heading for section}
Text for this sub-sub-heading \ldots
\paragraph*{Sub-sub-sub heading for section}
Text for this sub-sub-sub-heading \ldots
In this section we examine the growth rate of the mean of $Z_0$, $Z_1$ and $Z_2$. In
addition, we examine a common modeling assumption and note the
importance of considering the tails of the extinction time $T_x$ in
studies of escape dynamics.
We will first consider the expected resistant population at $vT_x$ for
some $v>0$, (and temporarily assume $\alpha=0$)
%
\[
 E \bigl[Z_1(vT_x) \bigr]= E
\biggl[\mu T_x\int_0^{v\wedge
1}Z_0(uT_x)
\exp \bigl(\lambda_1T_x(v-u) \bigr)\,du \biggr].
\]
%
If we assume that sensitive cells follow a deterministic decay
$Z_0(t)=xe^{\lambda_0 t}$ and approximate their extinction time as
$T_x\approx-\frac{1}{\lambda_0}\log x$, then we can heuristically
estimate the expected value as
%
\begin{eqnarray}\label{eqexpmuts}
E\bigl[Z_1(vT_x)\bigr] &=& \frac{\mu}{r}\log x
\int_0^{v\wedge1}x^{1-u}x^{({\lambda_1}/{r})(v-u)}\,du
\nonumber\\
&=& \frac{\mu}{r}x^{1-{\lambda_1}/{\lambda_0}v}\log x\int_0^{v\wedge
1}x^{-u(1+{\lambda_1}/{r})}\,du
\nonumber\\
&=& \frac{\mu}{\lambda_1-\lambda_0}x^{1+{\lambda_1}/{r}v} \biggl(1-\exp \biggl[-(v\wedge1) \biggl(1+
\frac{\lambda_1}{r}\biggr)\log x \biggr] \biggr).
\end{eqnarray}
%
Thus we observe that this expected value is finite for all $v>0$ (also see \cite{koon,khar,zvai,xjon,marg}).
%\nocite{oreg,schn,pond,smith,marg,hunn,advi,koha,mouse}

%%%%%%%%%%%%%%%%%%%%%%%%%%%%%%%%%%%%%%%%%%%%%%
%%                                          %%
%% Backmatter begins here                   %%
%%                                          %%
%%%%%%%%%%%%%%%%%%%%%%%%%%%%%%%%%%%%%%%%%%%%%%

\begin{backmatter}

\section*{Competing interests}
  The authors declare that they have no competing interests.

\section*{Author's contributions}
    Text for this section \ldots

\section*{Acknowledgements}
  Text for this section \ldots
%%%%%%%%%%%%%%%%%%%%%%%%%%%%%%%%%%%%%%%%%%%%%%%%%%%%%%%%%%%%%
%%                  The Bibliography                       %%
%%                                                         %%
%%  Bmc_mathpys.bst  will be used to                       %%
%%  create a .BBL file for submission.                     %%
%%  After submission of the .TEX file,                     %%
%%  you will be prompted to submit your .BBL file.         %%
%%                                                         %%
%%                                                         %%
%%  Note that the displayed Bibliography will not          %%
%%  necessarily be rendered by Latex exactly as specified  %%
%%  in the online Instructions for Authors.                %%
%%                                                         %%
%%%%%%%%%%%%%%%%%%%%%%%%%%%%%%%%%%%%%%%%%%%%%%%%%%%%%%%%%%%%%

% if your bibliography is in bibtex format, use those commands:
\bibliographystyle{bmc-mathphys} % Style BST file (bmc-mathphys, vancouver, spbasic).
\bibliography{zotero}      % Bibliography file (usually '*.bib' )
% for author-year bibliography (bmc-mathphys or spbasic)
% a) write to bib file (bmc-mathphys only)
% @settings{label, options="nameyear"}
% b) uncomment next line
%\nocite{label}

% or include bibliography directly:
% \begin{thebibliography}
% \bibitem{b1}
% \end{thebibliography}

%%%%%%%%%%%%%%%%%%%%%%%%%%%%%%%%%%%
%%                               %%
%% Figures                       %%
%%                               %%
%% NB: this is for captions and  %%
%% Titles. All graphics must be  %%
%% submitted separately and NOT  %%
%% included in the Tex document  %%
%%                               %%
%%%%%%%%%%%%%%%%%%%%%%%%%%%%%%%%%%%

%%
%% Do not use \listoffigures as most will included as separate files

%% \section*{Figures}
%%   \begin{figure}[h!]
%%   \caption{\csentence{Sample figure title.}
%%       A short description of the figure content
%%       should go here.}
%%       \end{figure}
%% 
%% \begin{figure}[h!]
%%   \caption{\csentence{Sample figure title.}
%%       Figure legend text.}
%%       \end{figure}

%%%%%%%%%%%%%%%%%%%%%%%%%%%%%%%%%%%
%%                               %%
%% Tables                        %%
%%                               %%
%%%%%%%%%%%%%%%%%%%%%%%%%%%%%%%%%%%

%% Use of \listoftables is discouraged.
%%
%% \section*{Tables}
%% \begin{table}[h!]
%% \caption{Sample table title. This is where the description of the table should go.}
%%       \begin{tabular}{cccc}
%%         \hline
%%            & B1  &B2   & B3\\ \hline
%%         A1 & 0.1 & 0.2 & 0.3\\
%%         A2 & ... & ..  & .\\
%%         A3 & ..  & .   & .\\ \hline
%%       \end{tabular}
%% \end{table}

%%%%%%%%%%%%%%%%%%%%%%%%%%%%%%%%%%%
%%                               %%
%% Additional Files              %%
%%                               %%
%%%%%%%%%%%%%%%%%%%%%%%%%%%%%%%%%%%

%% \section*{Additional Files}
%%   \subsection*{Additional file 1 --- Sample additional file title}
%%     Additional file descriptions text (including details of how to
%%     view the file, if it is in a non-standard format or the file extension).  This might
%%     refer to a multi-page table or a figure.
%% 
%%   \subsection*{Additional file 2 --- Sample additional file title}
%%     Additional file descriptions text.


\end{backmatter}
\end{document}
